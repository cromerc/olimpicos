\documentclass[12pt]{article}

\usepackage[utf8]{inputenc}
\usepackage[spanish]{babel}
\usepackage{array}
\usepackage{adjustbox}
\usepackage[left=2cm,right=2cm,top=2cm,bottom=2cm,a4paper]{geometry}
\usepackage{amsmath}
\usepackage{verbatim}
\usepackage{courier}

\usepackage{listings}
\usepackage{xcolor}

\lstdefinestyle{freefempp}{
	language=SQL,
	basicstyle=\scriptsize\ttfamily,
	commentstyle=\itshape\color{gray},
	keywordstyle=\color{blue},
	numberstyle=\color{red},
	stringstyle=\color{red},
	identifierstyle=\color{violet},
	showstringspaces=false,
	%float,
	%frame=single,
	%numbers=left
}

\lstset{
	basicstyle=\footnotesize\ttfamily,
	columns=flexible,
	breaklines=true,
	inputencoding=utf8,
	extendedchars=true,
	literate={á}{{\'a}}1 {é}{{\'e}}1 {í}{{\'i}}1 {ó}{{\'o}}1 {ú}{{\'u}}1 {ñ}{{\~n}}1 {Á}{{\'A}}1 {É}{{\'E}}1 {Í}{{\'I}}1 {Ó}{{\'O}}1 {Ú}{{\'U}}1 {Ñ}{{\~N}}1,
	style=freefempp
}

\usepackage{titling}
\renewcommand\maketitlehooka{\null\mbox{}\vfill}
\renewcommand\maketitlehookd{\vfill\null}

\usepackage{hyperref}
\hypersetup{colorlinks=true,allcolors=blue}
\usepackage{hypcap}

\title{\textbf{Trabajo Semestral de Administración y Programación de Base de Datos\\Entrega 1}}
\author{Christopher Cromer\\Ignacio Ortiz\\\\Profesoras Tatiana Gutiérrez y Valeria Beratto\\Ingeniería Civil Informática}
\date{\today}

\pretitle{%
	\begin{center}
		\LARGE
		\includegraphics[width=4cm]{ubblogo.png}\\[\bigskipamount]
	}

\posttitle{\end{center}}

\begin{document}
\clearpage
\hypersetup{pageanchor=false}
\begin{titlingpage}
\maketitle
\end{titlingpage}
\thispagestyle{empty}

\newpage

\clearpage
\thispagestyle{empty}
\tableofcontents

\newpage
\hypersetup{pageanchor=true}
\pagenumbering{arabic}
\section{Introducción}
Por el proyecto semestral se pide desarrollar un base de datos para los juegos olímpicos.\
El proyecto está dividido en 2 partes grandes, el modelo y el SQL.\
En el caso de \nameref{sec:modelo}, tenemos el \nameref{sec:mr} que fue usado para construir nuestro base de datos.
En la sección de \nameref{sec:sql} tenemos la \nameref{sec:schema}, que tiene la esquema del base de datos que hicimos basado en el\
modelo. Las \nameref{sec:vistas}, la única vista que tenemos es necesario para la funcionamiento de consulta 2.\
Creamos los \nameref{sec:triggers} 3 a 5 antes que poblamos el base de datos, la razón siendo que nuestros datos están
conforme con las reglas de los triggers. La sección de \nameref{sec:consultas} tiene las consultas 1 y 2. Y finalmente las\
\nameref{sec:pruebas} ayuda en demostrar que los triggers funcionan sin problema.

\newpage
\section{Modelo}
\label{sec:modelo}
\subsection{Modelo Relacional}
\label{sec:mr}
\begin{center}
    En negrita se presentan las claves foráneas y las primarias en subrayado.\smallskip
\end{center}\bigskip
{\scriptsize
\noindent
MEDALLA (\underline{ID\_MEDALLA}, DESCRIPCION)\bigskip

\noindent
CATEGORIA (\underline{ID\_CATEGORIA}, DESCRIPCION)\bigskip

\noindent
ATLETA (\underline{ID\_ATLETA}, NOMBRE, APELLIDO, FECHA\_NACIMIENTO, NACIONALIDAD)\bigskip

\noindent
PAIS (\underline{ID\_PAIS}, NOMBRE)\bigskip

\noindent
TIPOS\_DEPORTES (\underline{ID\_TIPO}, DESCRIPCION)\bigskip

\noindent
CIUDAD \underline{(ID\_CIUDAD}, NOMBRE, \textbf{PAIS})\bigskip

\noindent
LUGAR \underline{(ID\_LUGAR}, NOMBRE, CAPACIDAD, \textbf{CIUDAD})\bigskip

\noindent
HOTEL (\underline{ID\_HOTEL}, NOMBRE, \textbf{CIUDAD})\bigskip

\noindent
CAPITAL (\underline{\textbf{ID\_PAIS}, \textbf{ID\_CIUDAD}})\bigskip

\noindent
DELEGACION (\underline{ID\_DELEGACION}, NOMBRE, CANTIDAD\_ATLETAS, \textbf{PAIS}, \textbf{ATLETA})\bigskip

\noindent
COMPETENCIA (\underline{ID\_COMPETENCIA}, NOMBRE, \textbf{CATEGORIA}, \textbf{TIPO}, \textbf{LUGAR})\bigskip

\noindent
PRECOMPETENCIA (\underline{ID\_PRECOMPETENCIA}, FECHA, \textbf{COMPETENCIA}, \textbf{LUGAR})\bigskip

\noindent
HOSPEDAR (\underline{\textbf{ID\_HOTEL}, \textbf{ID\_DELEGACION}, FECHA\_INICIO}, FECHA\_FIN, PRECIO\_DELEGACION, PRECIO\_ATLETA)\bigskip

\noindent
GANAR (\underline{\textbf{ID\_ATLETA}, \textbf{ID\_COMPETENCIA}, FECHA}, \textbf{ID\_MEDALLA}, PUNTAJE)\bigskip

\noindent
PARTICIPAR\_COMPETENCIA (\underline{\textbf{ID\_COMPETENCIA}, \textbf{ID\_ATLETA}, FECHA})\bigskip

\noindent
PARTICIPAR\_PRECOMPETENCIA (\underline{\textbf{ID\_PRECOMPETENCIA}, \textbf{ID\_ATLETA}, FECHA}, PUNTAJE)
}

\newpage
\section{SQL}
\label{sec:sql}
\subsection{Schema}
\label{sec:schema}
Eso es la esquema general de nuestro proyecto de base de datos.\\
\lstinputlisting{../schema.sql}

\newpage
\subsection{Vistas}
\label{sec:vistas}
La siguiente vista es parte de consulta 2 en \nameref{sec:consultas}.\
Es necesario ser creado antes que se puede ejecutar la consulta.\\
\lstinputlisting{../vista.sql}

\newpage
\subsection{Triggers}
\label{sec:triggers}
Creamos los triggers antes que insertamos los datos de prueba.\
Eso garantiza que nuestros datos conforme con las reglas de los trigger.\\
\lstinputlisting{../trigger.sql}

\newpage
\subsection{Datos}
\label{sec:datos}
Los datos que usamos para popular el base de datos nos facilita en consulta 1 a 2 y trigger 3 a 5.
\lstinputlisting{../data.sql}

\newpage
\subsection{Consultas}
\label{sec:consultas}
Esos son las consultas 1 y 2. En el caso de consulta 2, depende de la vista creada anteriormente de\
la sección \nameref{sec:vistas}.
\lstinputlisting{../consulta.sql}

\newpage
\subsection{Pruebas para los Trigger}
\label{sec:pruebas}
Las consultas siguientes se pueden ejecutar para probar la funcionamiento correcto de los triggers.\\
\lstinputlisting{../trigger_test.sql}

\end{document}

